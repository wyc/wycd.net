\documentclass[12pt]{article}
\bibliographystyle{plain}

\usepackage{hyperref}
\usepackage{amsmath}

\title{Notes on Jlang}

\author{wycd.net}

\date{January, 2014}

\begin{document}
\maketitle

\begin{abstract}
This is a collection of non-comprehensive notes by a programmer learning the
J programming language who was previously familiar with C and C++. These
notes are meant to provide clarification for language features with examples.
If desired, please email any errors or ambiguities to \texttt{mail@wycd.net}.

\mbox{}\\
\texttt{j64-701}
\end{abstract}

\tableofcontents
\clearpage

\section{Syntax}
\subsection{Arithmetic}
\begin{tabular}{|l|l|l|}
\hline
Operation & Expression & J Expression \\
\hline
Addition & ($x + y$) & \texttt{x + y} \\
Multiplication & ($x * y$) & \texttt{x * y} \\ 
Subtraction & ($x - y$) & \texttt{x - y} \\
Division & ($\frac{x}{y}$) & \texttt{x \% y} \\
Modulo & ($x\ mod\ y$) & \texttt{y | x} \\
Reciprocal & ($x^{-1}$) & \texttt{\% x} \\
Negation & ($-x$) & \texttt{- x} \\
Power & ($x^y$) & \texttt{x \^{} y} \\
Squared & ($x^2$) & \texttt{*: x} \\
Square Root & ($\sqrt{x}$) & \texttt{\%: x} \\
Double & ($2x$) & \texttt{+: x} \\
Halve  &($\frac{1}{2}x$) & \texttt{-: x} \\
Signum & ($sgn(x)$) & \texttt{* x} \\
Exponential & ($e^{x}$) & \texttt{\^{} x} \\
Natural Log & ($ln(x)$) & \texttt{\^{}. x} \\
Log & ($log_y(x)$) & \texttt{y \^{}. x} \\
Pi Times & ($\pi x$) & \texttt{o. x} \\
Factorial & ($x!$) & \texttt{! x} \\
Floor & ($\lfloor x \rfloor$) & \texttt{<. x} \\
Ceiling & ($\lceil x \rceil$) & \texttt{>. x} \\
Sine  &($sin(x)$) & \texttt{1 o. x} \\
Cosine & ($cos(x)$) & \texttt{2 o. x} \\
Tangent & ($tan(x)$) & \texttt{3 o. x} \\
\hline
Complex (Dyadic) & ($Complex(\{x, y\}) \implies x + yi$) & \texttt{x j. y} \\
Complex (Monadic) & ($Complex(x) \implies xi$) & \texttt{j. x} \\
Magnitude, Angle & ($\{|x|, \angle x\}$) & \texttt{*. x} \\
Real, Imaginary & ($\{Re\{x\}, Im\{x\}\}$) & \texttt{+. x} \\
Conjugate & ($x^{*}$) & \texttt{+ x} \\
Magnitude & ($|x|$) & \texttt{| x} \\
\hline
\end{tabular}

\bigskip

Negative numbers are represented with a leading underscore: e.g. \texttt{\_42}
This allows for discrimination between the negative sign of a literal number
and the operator \texttt{-}.

\bigskip

Complex numbers are expressed in the form \texttt{xjy}, where $x$ is the real
part and $y$ is the imaginary in $x + yi$. The $j$ notation was most likely
chosen as not to conflict with the \texttt{i.} integer operator, but it is
convenient that this is the same notation used in electrical engineering.
$3 + 5i$ may be expressed as \texttt{3j5}.

\subsection{Evaluation Order}
Unless parentheses are used, J always evaluates from \textbf{right to left}.
The following are the same:
\begin{verbatim}
2 * 8 + 4 % 3 - 2
2 * (8 + (4 % (3 - 2)))
\end{verbatim}

\subsection{Monads and Dyads}
The same symbol may have different meanings depending on the context. For
example, the symbol \texttt{-} may be used for subtraction if placed between
two numbers (\texttt{4 - 2}), but it may also be used for negation if there
is only one number (\texttt{-4}). A function with a single argument on its
right is called a monadic function (monad) while a function taking one argument
on each side is called a dyadic function (dyad). The monad \texttt{- x} has
different functionality than the dyad \texttt{x - y}.

\subsection{Types}
Thre are four primitive types in J: \textbf{number, character, box, and symbol}.
There are six subtypes of number two subtypes of character plus there are
sparse versions of some of the types. Boxes are general-purpose containers
while symbols consist of the operators in the J language.

The full list can be found at
\url{http://www.jsoftware.com/docs/help701/dictionary/dx003.htm}

\subsection{Parts of Speech}
Functions in J are known as \textbf{verbs} because they operate on expressions
which are known as \textbf{nouns}. For example, in the expression
\texttt{2 + 3}, the verb \texttt{+} dyadically operates on the nouns \texttt{2}
and \texttt{3} to yield the noun \texttt{5}. In \texttt{*: 8}, the square verb
\texttt{*:} operates on the single noun \texttt{8} to yield the noun 
texttt{64}.

\paragraph{}
\texttt{Adverbs} are operators that act upon and modify the behavior of
functions just as adverbs in English affect verbs. For example, the Insert
adverb \texttt{/} causes a verb to be inserted between all elements in a
list. These are the same:
\begin{verbatim}
+/ 1 2 3 4 5
1 + 2 + 3 + 4 + 5
\end{verbatim}
The new verb \texttt{+/} is formed by the verb and adverb, and might be
labeled \texttt{sum =: +/} for convenience. Adverbs always 




\subsection{Notation}
Throughout the J literature,
\begin{itemize}
\item Noun arguments to a verb are \texttt{x} and \texttt{y}
     (e.g. \texttt{x + y}).
\item Verb arguments to a conjunction are \texttt{u} and \texttt{v}
     (e.g. \texttt{u @ v}).
\item Verbs can also be \texttt{f}, \texttt{g}, and \texttt{h}
     (e.g. \texttt{x (f g h) y}).
\item Noun arguments to a conjunction are \texttt{m} and \texttt{n}
     (e.g. \texttt{m " n}).
\end{itemize}

\subsection{Arrays and Operations}
The name \texttt{array} in J refers to a data object with a number of
dimensions. Scalars (0-dimensional), lists (1-dimensional) and tables
(2-dimensional) are all considered arrays. Scalars can be expressed by
single terms such as \texttt{5}, \texttt{3j5}, or \texttt{1.25}. Lists
can be constructed from adjacent scalars such as \texttt{1 2 3 4 5},
\texttt{1.1 1.25 3j3 1 3 9 7}, or \texttt{'abcde'}. 

The Shape verb \texttt{\$} can be used to manipulate 

\clearpage

\nocite{*}
\bibliography{references}

\end{document}

