\documentclass[12pt]{article}
\bibliographystyle{plain}

\usepackage{hyperref}

\title{Notes on Golang}

\author{wycd.net}

\date{January, 2014}

\begin{document}
\maketitle

\begin{abstract}
This is a collection of non-comprehensive notes by a programmer learning the
Go programming language who was previously familiar with C and C++. These
notes are meant to provide clarification for language features with examples.
Although the Golang specs are already very clear, it could prove helpful to
have additional examples of usage. If desired, please email any errors or
ambiguities to \texttt{mail@wycd.net}.

\mbox{}\\
\texttt{go version go1.1.2 linux/amd64}
\end{abstract}

\tableofcontents
\clearpage

\section{Syntax}
\subsection{Variables}
\begin{itemize}
    \item All variables are implicitly initialized to their zero-value.

    \url{http://golang.org/ref/spec#The_zero_value}
    \item No implicit conversions. \texttt{T(v)} must be used to convert
          the value \texttt{v} to the type \texttt{T}:
        \begin{verbatim}
var i int = 42
var f float64 = float64(i)
var u uint = uint(f)
        \end{verbatim}
    \item These are the same:
        \begin{verbatim}
var foo string = "text"
var foo = "text"
foo := "text"
        \end{verbatim}
    \item All types implement the empty interface \texttt{interface\{\}}
    \item Type assertion may be used \texttt{x.(T)} to verify an expression
          \texttt{x} of \texttt{interface\{\}} type is not \texttt{nil} and
          stores a value of type \texttt{T}.
    \item Type inference with decimal numbers results in \texttt{float64}.
        \begin{verbatim}
var t interface{} = 1.0 /* for type assertion */
f64 := t.(float64)      /* okay */
f32 := t.(float32)      /* error */
        \end{verbatim}
    \item Selectors dereference pointers to structs automatically.
        \begin{verbatim}
type Person struct { name string }

var (
    o = Person{name: "Fred"}
    p = &o
)

func main() {
    /* the following statements have the same effect */
    o.name = "George"
    p.name = "George"
}

        \end{verbatim}

    \item structs may have anonymous fields and can then satisfy methods
          that have receivers for those field types. The field is
          \textit{promoted} and can by accessed by a selector with that
          type's name.
        \begin{verbatim}
package main

import "fmt"

type Int int

type IntWrap struct {
        Int
}

func (i Int) test() {
        fmt.Println("test okay:", i)
}

func main() {
        a := IntWrap{42} 
        a.test()
    fmt.Println("tested value:", a.Int)
}
        \end{verbatim}
          
    \item The \texttt{new(T)} allocates space for a T value and
          returns its address.
          \begin{verbatim}
var i *int = new(int)
fmt.Println(*i)
          \end{verbatim}

    \item Slices (e.g. \texttt{[]int}) are wrappers around arrays
          containing metadata, allowing for convenient features
          such as (fake) dynamic sizing. They must be created with
          \texttt{make()}. See \\
          \url{http://blog.golang.org/go-slices-usage-and-internals}.

          \begin{verbatim}
var s []int = make([]int, 5, 10) /* s is [0 0 0 0 0] */
len(s) /* 5 */
cap(s) /* 10 */

s = s[:cap(s)] /* s is [0 0 0 0 0 0 0 0 0 0] */
len(s) /* 10 */
cap(s) /* 10 */

s = s[2:] /* s is [0 0 0 0 0 0 0 0]                  */
          /* elements 0 and 1 are now "lost forever" */
          /* ...no such thing as s[-2:]              */
          \end{verbatim}
    \item Maps are created with \texttt{make()} and work like maps
        in any other language. They can return an optional value \texttt{ok}
        which is \texttt{true} if the value exists and \texttt{false} if not.
        If the value does not exist, the type's default \texttt{zero value}
        is returned as the value. In gc version 6g, maps are implemented as
        hashmaps: \url{http://golang.org/src/pkg/runtime/hashmap.c}.
        \begin{verbatim}
m := make(map[string]int)
m["hello"] = 4
value := m["hello"] /* value is 4 */
value, ok := m["hello"] /* value is 4, ok is true */
value, ok = m["cheese"] /* value is 0, ok is false */
\end{verbatim}
    \item Channels work like pipes. They behave like FIFO data structures
          and are safe to use concurrently. Data goes in direction of the
          \texttt{<-} arrows. They will block on send or receive by default.
        \begin{verbatim}
ic := make(chan int)      /* make a channel that can pass ints */
go func() { ic <- 5 }()   /* send 5 into ic (blocking until received) */
val := <- ic              /* receive the 5 from ic into val */
        \end{verbatim}
    \item Channels may be created with a buffer size as a second argument
          to the \texttt{make()} function. The default buffer size is zero,
          causing channel senders to block until a channel receiver dequeues
          the value. With buffers, queueing without a channel receiver becomes
          a non-blocking operation as long as the channel is not full.
        \begin{verbatim}
ic := make(chan int, 100) /* make buffered channel that can pass ints */
ic <- 5                   /* send 5 into ic (non-blocking due to buffer) */
val := <- ic              /* receive the 5 from ic into val */
        \end{verbatim}
\end{itemize}


\subsection{Control Flow}
\begin{itemize}
    \item \texttt{if...else} statements may have a initializing
        statement, and they require braces but not parentheses.
      \begin{verbatim}
if f := 1; f < 3 {
    ...
} else {
    ...
}
          \end{verbatim}
    \item The \texttt{range} clause of the for loop allows for iteration
          over slices or maps.
          
          \begin{tabular}{| l | l | l | l |}
          \hline
          \texttt{Range Expression} &
          \texttt{Literal Expression} &
          \texttt{1st value} &
          \texttt{2nd value} \\
          \hline
          \texttt{array or slice} &
          \texttt{a  [n]E, *[n]E, or []E} &
          \texttt{index    i  int} &
          \texttt{a[i]       E} \\
          \hline
          \texttt{string} &
          \texttt{s  string type} &
          \texttt{index    i  int} &
          \texttt{r rune} \\
          \hline
          \texttt{map} &
          \texttt{m  map[K]V} &
          \texttt{key      k  K} &
          \texttt{m[k]       V} \\
          \hline
          \texttt{channel} &
          \texttt{c  chan E, <-chan E} &
          \texttt{element  e  E} & \\
          \hline
          \end{tabular}
          \bigskip

          More information and above table source here:
          \url{http://golang.org/ref/spec#For_statements}
        \begin{verbatim}
shifts := []int{1, 2, 4, 8, 16, 32}
for i, v := range shifts { /* i is optional */
    fmt.Printf("1 << %d = %d\n", i, v)
}
        \end{verbatim}
    \item The \texttt{switch} statement automatically terminates at the end
    of a case, but may fallthrough to subsequent cases using the
    \texttt{fallthrough} keyword. Case statements evaluate expressions if not
    given a variable to match against.
        \begin{verbatim}
v := 5
switch { /* if...else style-usage */
case v%2 == 0:
    fmt.Println("even")
default:
    fmt.Println("odd")
}

c := ' '
switch c { /* match-style usage */
case ' ', '\t', '\n', '\r':
    fmt.Println("space")
default:
    fmt.Println("not a space")
    
}
        \end{verbatim}
    \item The \texttt{select} statement is structured like the switch
        statement, but has the main functionality of triggering depending
        on what \texttt{channel}s are available.
        \begin{verbatim}
var c1, c2 chan int
select {
case c1 <- (<-c2):
        fmt.Println("popped a value from c2 into c1")
        break
default:
        fmt.Println("either c2 or c1 not ready")
}
        \end{verbatim}
\end{itemize}
    
\subsection{Functions}
\begin{itemize}
    \item These are the same:
        \begin{verbatim}
func add(x int, y int) int {...} 
func add(x, y int) int {...}
        \end{verbatim}
    \item Functions can return multiple values:
        \begin{verbatim}
func beginningMiddleEnd(x float64) (float64, float64, float64) {
    return 0.0, (x / 2), x
}

func main() {
    b, m, e := beginningMiddleEnd(12)
}
        \end{verbatim}
        Note that these return values do not form a first-class object,
        so they may not be accessed as members (i.e. \texttt{f()[0]}).
    \item Named return variables allow for implicit return values.
        \begin{verbatim}
func polarToCartesian(r, theta float64) (x, y float64) {
    x = r * math.Cos(theta)
    y = r * math.Sin(theta)
    return
}
        \end{verbatim}

    \item Functions may be assigned to variables because they are values.
        \begin{verbatim}
var inc func(int) int = func(x int) int {
    return x + 1
}
        \end{verbatim}
    \item A function may be a closure and reference variables outside of its
    body. 
        \begin{verbatim}
func accumulator() func(int) []int {
    s := make([]int, 0)
    return func(x int) []int {
        s = append(s, x)
        return s
    }
}

func main() {
    var acc func(int) []int = accumulator()
    acc(4)
    acc(3)
    acc(2)
    acc(1)
    acc(0) /* expression is [4 3 2 1 0] */
}
        \end{verbatim}

    \item goroutines allow for concurrency. It is used as \texttt{go f()}
        to run \texttt{f()} as a parallel goroutine. Channels are commonly
        used to control goroutines, but Go also provides locks and wait groups
        in standard \texttt{sync} package. However, there are currently not
        provided methods similar to \texttt{join} for threads. The function
        call may be a closure:
        \begin{verbatim}
package main

import "fmt"

func main() {
    done := make(chan bool)
    go func() { fmt.Println("test"); done <- true }()
    <-done
}
        \end{verbatim}

\end{itemize}

\subsection{Methods and Interfaces}
\begin{itemize}
    \item Methods are ``glued'' on to struct types.
        \begin{verbatim}
type Rectangle struct {
    W, H float64
}

func (r *Rectangle) Area() float64 {
    return r.W * r.H
}

func main() {
    var r *Rectangle = &Rectangle{8, 10}
    var area float64 = r.Area()
}
        \end{verbatim}
    \item In the above function definition for \texttt{Area()}, the text
          \texttt{(r *Rectangle)} represents a \textit{receiver} for the
          type \texttt{*Rectangle}. Depending on if the type is
          \texttt{*Rectangle} or \texttt{Rectangle}, the 
          method receives a pointer to the original or a new copy,
          respectively. 
    \item Interfaces specify methods that must be implemented to be
          considered that interface. ``No explicit declaration of
          intent'' is required, and a type implements an interface
          just by having all the required methods.
        \begin{verbatim}
type Shape interface {
    Area() float64
}

type Rectangle struct {
    W, H float64
}

func (r *Rectangle) Area() float64 {
    return r.W * r.H
}

func main() {
    var s Shape = &Rectangle{5, 10} /* this is fine */
}
        \end{verbatim}
    \item In the above example, a method with the receiver
        \texttt{(r Rectangle)} would have also sufficed because of
        method sets. A pointer type \texttt{*T} has methods with
        receivers of both \texttt{*T} and \texttt{T} in its method
        set. 
        
        \url{http://golang.org/ref/spec#Method_sets}
\end{itemize}

\section{Packages}
\subsection{\texttt{fmt}}
\begin{itemize}
    \item \texttt{fmt} printing methods can print values of primitives
        and try to use reflection to call the \texttt{String()} routine
        of an object. They will otherwise try to print the object's address
        in \texttt{\&\{0xdeadbeef\}} form. See \texttt{printArg()} in
        \texttt{fmt/print.go} for details.
    \item \texttt{fmt.Println(arg0, arg1, ...)} automatically adds a
        space between arguments when printing.
\end{itemize}

\subsection{\texttt{errors}}
\begin{itemize}
    \item The built-in interface type \texttt{error} only has one method
        \texttt{Error() string}. Anything satisfying this implements the
        \texttt{error} interface.
\end{itemize}

\clearpage

\nocite{*}
\bibliography{references}

\end{document}

