\documentclass[12pt]{article}
\bibliographystyle{plain}

\usepackage{hyperref}
\usepackage{amsfonts}
\usepackage{amsmath}
\usepackage{accents}
\usepackage{xcolor}
\usepackage{alltt}

\title{Notes on Probability}

\author{wycd.net}

\date{Spring, 2014}

\begin{document}
\maketitle

\tableofcontents
\clearpage

\section{Definitions}
A \textbf{sample space} is the set of all possible outcomes of an experiment.

An \textbf{event} is a subset of the sample space.

\textbf{Naive definition of probability:}

$P(A) = \frac{\text{# favorable outcomes}}{\text{# possible outcomes}}$

Assumes all outcomes equally likely, finite sample space.

\subsection{Counting}
Multiplication Rule: there is an experiment with $n_1$ possible outcomes, and
for each outcome of the first expression there are $n_2$ outcomes for the 2nd
expresssion, ...., for each there are $n_r$ outcomes for the $r$th expression,
then there are $n_1n_2...n_r$ overall possible outcomes.

Binomial Coefficient: $n \choose k = \frac{n!}{(n-k)!k!}; 0 if k > n$

Probability of a full house in poker: $\frac{13 \times 4 \choose 3 \times 12 \times 4 \choose 2}{52 \choose 5}$

Sampling table: choose $k$ objects out of $n$.
Replace, Order Matters: $n^k$
Replace, Order Doesn't Matter: ${n + k -1} \choose k$
Don't Replace, Order Matters: $n(n-1)...(n-k+1)$
Don't Replace, Order Doesn't Matter: $n \choose k$

Replace, Order Doesn't Matter : how many ways are there to put $k$ indistinguishable particles into $n$ distinguishable boxes?

Imagine separators. $n+k-1$ positions, choose $k$ or $n-1$

Story Proof:

Proof by interpretation.

$n{n-1} \choose {k-1} = k{n} \choose {k}$

Pick $k$ people out of $n$, with one designated as the president.

${m+n} \choose {k} = \sum_{j=0}^{k}{m \choose k}{n \choose k - j}$ (Vandermonde)

\section{Probability}
Non-naived definition: A \textbf{probability sample} consists of $S$ and $P$, where $S$ is a sample space, and $P$ is a function which takes an event $A \subseteq S$ as input, and return $P(A) \in [0, 1]$ as outputs.

Such that

(1) $P(\phi) = 0, P(S) = 1$

(2) $P(U^{\infty}_{n=1}A_n) = \sum_{n=1}{\infty} P(A_n) if \forall i A_i is disjoint$

Properties:

(1) $P(A^c) = 1 - P(A)$
\[P(A^c + A)\]
(2) If $A \subseteq B$, then $P(A) \le P(B)$.
\[B = A \cup (B \cap A^c)\]

(3) $P(A \cup B) = P(A) + P(B) - P(A \cap B)$
\[P(A \cup B) = P(A \cup (B \cap A^c))\]
\[= P(A) + P(B \cap A^C)\]
\[\stackrel{?}{=}P(A) + P(B) - P(A \cap B)\]
\[P(A\cap B) + P(B\cap A^c) = P(B)\]


Birthday Problem:

Use stars and bars.

n - 1
n - k + 1

(Inclusion-Exclusion)

$P(A\cup B\cup C) = P(A) + P(B) + P(C) - P(A\cap B) - P(A \cap C) - P(B \cap C) + P(A\cap B\cap C)$

$P(A\cup A_2\cup ...\cup A_n) = \sum_{j = 1}^n - \sum_{i<j} P(A_i \cap A_j) + \sum_{}$

deMontmort's Problem, matching

n cards labeled 1, 2, ..., n. Let $A_j$ be the event ``jth card matches''

$P(A_j) = \frac{1/n}$ since all positions are equally likely for card labeled j.

$P(A_1 \cap A_2) = \frac{(n-2)!}{n!} = \frac{1}{n(n-1)}$

$P(A_1 \cap ... \cap A_n) = \frac{(n-k)!}{k}$

\[(P(A_1\cap ...\cap A_k) = \frac{(n-k)!{n!}, there are {n \choose k} such terms\]

Independent if
\[P(A\cap B) = P(A\cup B)\]

A, B, C are independent if $P(A,B) = P(A)P(B), P(A,C) = P(A)P(C), P(B,C) = P(B)P(C), P(A,B,C)=P(A)P(B)P(C)$

Similarly for events $A_1,...,A_n$

Newton-Pepys  Problem (1693)

With fair dice, which is most likely?

(A) at least one 6 with 6 dice

(B) at least two 6 with 12 dice

(C) at least three 6 with 18 dice

\section{Conditional Probability}

$P(A_1\cup A_2\cup A_n) = $

$P(A|B) = \frac{P(A\capB)}{P(B)}$

Intuition I:
remove $B^c$ pebbles, renormalize to make mass 1 again

Intuition II:
frequentist world, repeat expt many times


Theroem I: $P(A \cap B) = P(B)P(A|B) = P(A)P(B|A)$
Theorem II: $P(A_1,...,A_n) = P(A_1)P(A_2k12k$



\end{document}

